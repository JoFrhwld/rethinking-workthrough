% Options for packages loaded elsewhere
\PassOptionsToPackage{unicode}{hyperref}
\PassOptionsToPackage{hyphens}{url}
\PassOptionsToPackage{dvipsnames,svgnames,x11names}{xcolor}
%
\documentclass[
  letterpaper,
  DIV=11,
  numbers=noendperiod]{scrartcl}

\usepackage{amsmath,amssymb}
\usepackage{iftex}
\ifPDFTeX
  \usepackage[T1]{fontenc}
  \usepackage[utf8]{inputenc}
  \usepackage{textcomp} % provide euro and other symbols
\else % if luatex or xetex
  \usepackage{unicode-math}
  \defaultfontfeatures{Scale=MatchLowercase}
  \defaultfontfeatures[\rmfamily]{Ligatures=TeX,Scale=1}
\fi
\usepackage{lmodern}
\ifPDFTeX\else  
    % xetex/luatex font selection
\fi
% Use upquote if available, for straight quotes in verbatim environments
\IfFileExists{upquote.sty}{\usepackage{upquote}}{}
\IfFileExists{microtype.sty}{% use microtype if available
  \usepackage[]{microtype}
  \UseMicrotypeSet[protrusion]{basicmath} % disable protrusion for tt fonts
}{}
\makeatletter
\@ifundefined{KOMAClassName}{% if non-KOMA class
  \IfFileExists{parskip.sty}{%
    \usepackage{parskip}
  }{% else
    \setlength{\parindent}{0pt}
    \setlength{\parskip}{6pt plus 2pt minus 1pt}}
}{% if KOMA class
  \KOMAoptions{parskip=half}}
\makeatother
\usepackage{xcolor}
\setlength{\emergencystretch}{3em} % prevent overfull lines
\setcounter{secnumdepth}{-\maxdimen} % remove section numbering
% Make \paragraph and \subparagraph free-standing
\ifx\paragraph\undefined\else
  \let\oldparagraph\paragraph
  \renewcommand{\paragraph}[1]{\oldparagraph{#1}\mbox{}}
\fi
\ifx\subparagraph\undefined\else
  \let\oldsubparagraph\subparagraph
  \renewcommand{\subparagraph}[1]{\oldsubparagraph{#1}\mbox{}}
\fi

\usepackage{color}
\usepackage{fancyvrb}
\newcommand{\VerbBar}{|}
\newcommand{\VERB}{\Verb[commandchars=\\\{\}]}
\DefineVerbatimEnvironment{Highlighting}{Verbatim}{commandchars=\\\{\}}
% Add ',fontsize=\small' for more characters per line
\usepackage{framed}
\definecolor{shadecolor}{RGB}{241,243,245}
\newenvironment{Shaded}{\begin{snugshade}}{\end{snugshade}}
\newcommand{\AlertTok}[1]{\textcolor[rgb]{0.68,0.00,0.00}{#1}}
\newcommand{\AnnotationTok}[1]{\textcolor[rgb]{0.37,0.37,0.37}{#1}}
\newcommand{\AttributeTok}[1]{\textcolor[rgb]{0.40,0.45,0.13}{#1}}
\newcommand{\BaseNTok}[1]{\textcolor[rgb]{0.68,0.00,0.00}{#1}}
\newcommand{\BuiltInTok}[1]{\textcolor[rgb]{0.00,0.23,0.31}{#1}}
\newcommand{\CharTok}[1]{\textcolor[rgb]{0.13,0.47,0.30}{#1}}
\newcommand{\CommentTok}[1]{\textcolor[rgb]{0.37,0.37,0.37}{#1}}
\newcommand{\CommentVarTok}[1]{\textcolor[rgb]{0.37,0.37,0.37}{\textit{#1}}}
\newcommand{\ConstantTok}[1]{\textcolor[rgb]{0.56,0.35,0.01}{#1}}
\newcommand{\ControlFlowTok}[1]{\textcolor[rgb]{0.00,0.23,0.31}{#1}}
\newcommand{\DataTypeTok}[1]{\textcolor[rgb]{0.68,0.00,0.00}{#1}}
\newcommand{\DecValTok}[1]{\textcolor[rgb]{0.68,0.00,0.00}{#1}}
\newcommand{\DocumentationTok}[1]{\textcolor[rgb]{0.37,0.37,0.37}{\textit{#1}}}
\newcommand{\ErrorTok}[1]{\textcolor[rgb]{0.68,0.00,0.00}{#1}}
\newcommand{\ExtensionTok}[1]{\textcolor[rgb]{0.00,0.23,0.31}{#1}}
\newcommand{\FloatTok}[1]{\textcolor[rgb]{0.68,0.00,0.00}{#1}}
\newcommand{\FunctionTok}[1]{\textcolor[rgb]{0.28,0.35,0.67}{#1}}
\newcommand{\ImportTok}[1]{\textcolor[rgb]{0.00,0.46,0.62}{#1}}
\newcommand{\InformationTok}[1]{\textcolor[rgb]{0.37,0.37,0.37}{#1}}
\newcommand{\KeywordTok}[1]{\textcolor[rgb]{0.00,0.23,0.31}{#1}}
\newcommand{\NormalTok}[1]{\textcolor[rgb]{0.00,0.23,0.31}{#1}}
\newcommand{\OperatorTok}[1]{\textcolor[rgb]{0.37,0.37,0.37}{#1}}
\newcommand{\OtherTok}[1]{\textcolor[rgb]{0.00,0.23,0.31}{#1}}
\newcommand{\PreprocessorTok}[1]{\textcolor[rgb]{0.68,0.00,0.00}{#1}}
\newcommand{\RegionMarkerTok}[1]{\textcolor[rgb]{0.00,0.23,0.31}{#1}}
\newcommand{\SpecialCharTok}[1]{\textcolor[rgb]{0.37,0.37,0.37}{#1}}
\newcommand{\SpecialStringTok}[1]{\textcolor[rgb]{0.13,0.47,0.30}{#1}}
\newcommand{\StringTok}[1]{\textcolor[rgb]{0.13,0.47,0.30}{#1}}
\newcommand{\VariableTok}[1]{\textcolor[rgb]{0.07,0.07,0.07}{#1}}
\newcommand{\VerbatimStringTok}[1]{\textcolor[rgb]{0.13,0.47,0.30}{#1}}
\newcommand{\WarningTok}[1]{\textcolor[rgb]{0.37,0.37,0.37}{\textit{#1}}}

\providecommand{\tightlist}{%
  \setlength{\itemsep}{0pt}\setlength{\parskip}{0pt}}\usepackage{longtable,booktabs,array}
\usepackage{calc} % for calculating minipage widths
% Correct order of tables after \paragraph or \subparagraph
\usepackage{etoolbox}
\makeatletter
\patchcmd\longtable{\par}{\if@noskipsec\mbox{}\fi\par}{}{}
\makeatother
% Allow footnotes in longtable head/foot
\IfFileExists{footnotehyper.sty}{\usepackage{footnotehyper}}{\usepackage{footnote}}
\makesavenoteenv{longtable}
\usepackage{graphicx}
\makeatletter
\def\maxwidth{\ifdim\Gin@nat@width>\linewidth\linewidth\else\Gin@nat@width\fi}
\def\maxheight{\ifdim\Gin@nat@height>\textheight\textheight\else\Gin@nat@height\fi}
\makeatother
% Scale images if necessary, so that they will not overflow the page
% margins by default, and it is still possible to overwrite the defaults
% using explicit options in \includegraphics[width, height, ...]{}
\setkeys{Gin}{width=\maxwidth,height=\maxheight,keepaspectratio}
% Set default figure placement to htbp
\makeatletter
\def\fps@figure{htbp}
\makeatother

\usepackage{booktabs}
\usepackage{caption}
\usepackage{longtable}
\KOMAoption{captions}{tableheading}
\makeatletter
\@ifpackageloaded{tcolorbox}{}{\usepackage[skins,breakable]{tcolorbox}}
\@ifpackageloaded{fontawesome5}{}{\usepackage{fontawesome5}}
\definecolor{quarto-callout-color}{HTML}{909090}
\definecolor{quarto-callout-note-color}{HTML}{0758E5}
\definecolor{quarto-callout-important-color}{HTML}{CC1914}
\definecolor{quarto-callout-warning-color}{HTML}{EB9113}
\definecolor{quarto-callout-tip-color}{HTML}{00A047}
\definecolor{quarto-callout-caution-color}{HTML}{FC5300}
\definecolor{quarto-callout-color-frame}{HTML}{acacac}
\definecolor{quarto-callout-note-color-frame}{HTML}{4582ec}
\definecolor{quarto-callout-important-color-frame}{HTML}{d9534f}
\definecolor{quarto-callout-warning-color-frame}{HTML}{f0ad4e}
\definecolor{quarto-callout-tip-color-frame}{HTML}{02b875}
\definecolor{quarto-callout-caution-color-frame}{HTML}{fd7e14}
\makeatother
\makeatletter
\makeatother
\makeatletter
\makeatother
\makeatletter
\@ifpackageloaded{caption}{}{\usepackage{caption}}
\AtBeginDocument{%
\ifdefined\contentsname
  \renewcommand*\contentsname{Table of contents}
\else
  \newcommand\contentsname{Table of contents}
\fi
\ifdefined\listfigurename
  \renewcommand*\listfigurename{List of Figures}
\else
  \newcommand\listfigurename{List of Figures}
\fi
\ifdefined\listtablename
  \renewcommand*\listtablename{List of Tables}
\else
  \newcommand\listtablename{List of Tables}
\fi
\ifdefined\figurename
  \renewcommand*\figurename{Figure}
\else
  \newcommand\figurename{Figure}
\fi
\ifdefined\tablename
  \renewcommand*\tablename{Table}
\else
  \newcommand\tablename{Table}
\fi
}
\@ifpackageloaded{float}{}{\usepackage{float}}
\floatstyle{ruled}
\@ifundefined{c@chapter}{\newfloat{codelisting}{h}{lop}}{\newfloat{codelisting}{h}{lop}[chapter]}
\floatname{codelisting}{Listing}
\newcommand*\listoflistings{\listof{codelisting}{List of Listings}}
\makeatother
\makeatletter
\@ifpackageloaded{caption}{}{\usepackage{caption}}
\@ifpackageloaded{subcaption}{}{\usepackage{subcaption}}
\makeatother
\makeatletter
\@ifpackageloaded{tcolorbox}{}{\usepackage[skins,breakable]{tcolorbox}}
\makeatother
\makeatletter
\@ifundefined{shadecolor}{\definecolor{shadecolor}{rgb}{.97, .97, .97}}
\makeatother
\makeatletter
\makeatother
\makeatletter
\makeatother
\ifLuaTeX
  \usepackage{selnolig}  % disable illegal ligatures
\fi
\IfFileExists{bookmark.sty}{\usepackage{bookmark}}{\usepackage{hyperref}}
\IfFileExists{xurl.sty}{\usepackage{xurl}}{} % add URL line breaks if available
\urlstyle{same} % disable monospaced font for URLs
\hypersetup{
  pdftitle={Garden of Forking paths part 2},
  pdfauthor={Josef Fruehwald},
  colorlinks=true,
  linkcolor={blue},
  filecolor={Maroon},
  citecolor={Blue},
  urlcolor={Blue},
  pdfcreator={LaTeX via pandoc}}

\title{Garden of Forking paths part 2}
\author{Josef Fruehwald}
\date{2023-05-10}

\begin{document}
\maketitle
\ifdefined\Shaded\renewenvironment{Shaded}{\begin{tcolorbox}[interior hidden, boxrule=0pt, borderline west={3pt}{0pt}{shadecolor}, frame hidden, breakable, enhanced, sharp corners]}{\end{tcolorbox}}\fi

\begin{tcolorbox}[enhanced jigsaw, breakable, opacityback=0, toprule=.15mm, colbacktitle=quarto-callout-note-color!10!white, colframe=quarto-callout-note-color-frame, rightrule=.15mm, colback=white, title=\textcolor{quarto-callout-note-color}{\faInfo}\hspace{0.5em}{Listening}, bottomrule=.15mm, coltitle=black, arc=.35mm, bottomtitle=1mm, toptitle=1mm, titlerule=0mm, leftrule=.75mm, left=2mm, opacitybacktitle=0.6]

\end{tcolorbox}

\begin{Shaded}
\begin{Highlighting}[]
\FunctionTok{library}\NormalTok{(tidyverse)}
\FunctionTok{library}\NormalTok{(gt)}
\FunctionTok{library}\NormalTok{(patchwork)}
\FunctionTok{library}\NormalTok{(here)}
\FunctionTok{source}\NormalTok{(}\FunctionTok{here}\NormalTok{(}\StringTok{"\_defaults.R"}\NormalTok{))}
\end{Highlighting}
\end{Shaded}

\hypertarget{a-nicer-table-version.}{%
\subsection{A nicer table version.}\label{a-nicer-table-version.}}

I'd like to re-represent the Bayesian Update in a nicer GT table. Some
options are

\begin{enumerate}
\def\labelenumi{\arabic{enumi}.}
\tightlist
\item
  Plotting extensions from \texttt{\{gtExtras\}}
\item
  Emojis
\end{enumerate}

\begin{Shaded}
\begin{Highlighting}[]
\NormalTok{renv}\SpecialCharTok{::}\FunctionTok{install}\NormalTok{(}\StringTok{"gtExtras"}\NormalTok{)}
\NormalTok{renv}\SpecialCharTok{::}\FunctionTok{install}\NormalTok{(}\StringTok{"svglite"}\NormalTok{)}
\NormalTok{renv}\SpecialCharTok{::}\FunctionTok{install}\NormalTok{(}\StringTok{"emoji"}\NormalTok{)}
\end{Highlighting}
\end{Shaded}

\begin{Shaded}
\begin{Highlighting}[]
\FunctionTok{library}\NormalTok{(gtExtras)}
\FunctionTok{library}\NormalTok{(emoji)}
\end{Highlighting}
\end{Shaded}

First, trying the ``win/losses'' column plot from \texttt{\{gtExtra\}}
to illustrate the blue vs white marbles.

\begin{Shaded}
\begin{Highlighting}[]
\FunctionTok{tibble}\NormalTok{(}
  \AttributeTok{blue\_marbs =} \DecValTok{0}\SpecialCharTok{:}\DecValTok{4}\NormalTok{,}
  \AttributeTok{white\_marbs =} \DecValTok{4} \SpecialCharTok{{-}}\NormalTok{ blue\_marbs}
\NormalTok{) }\SpecialCharTok{|\textgreater{}} 
  \FunctionTok{rowwise}\NormalTok{() }\SpecialCharTok{|\textgreater{}} 
  \FunctionTok{mutate}\NormalTok{(}
    \AttributeTok{marbles =} \FunctionTok{list}\NormalTok{(}\FunctionTok{c}\NormalTok{(}\FunctionTok{rep}\NormalTok{(}\DecValTok{1}\NormalTok{, blue\_marbs), }
                     \FunctionTok{rep}\NormalTok{(}\DecValTok{0}\NormalTok{, white\_marbs)))}
\NormalTok{  ) }\OtherTok{{-}\textgreater{}} 
\NormalTok{  marbles\_wl}
\end{Highlighting}
\end{Shaded}

The cell background will have to be off-white for the white ticks to
show

\begin{Shaded}
\begin{Highlighting}[]
\NormalTok{marbles\_wl }\SpecialCharTok{|\textgreater{}} 
  \FunctionTok{gt}\NormalTok{() }\SpecialCharTok{|\textgreater{}} 
  \FunctionTok{gt\_plt\_winloss}\NormalTok{(marbles, }\AttributeTok{palette =} \FunctionTok{c}\NormalTok{(}\StringTok{"blue"}\NormalTok{, }\StringTok{"white"}\NormalTok{, }\StringTok{"grey"}\NormalTok{)) }\SpecialCharTok{|\textgreater{}} 
  \FunctionTok{tab\_style}\NormalTok{(}\AttributeTok{style =} \FunctionTok{cell\_fill}\NormalTok{(}\AttributeTok{color =} \StringTok{"antiquewhite"}\NormalTok{), }
            \AttributeTok{locations =} \FunctionTok{cells\_body}\NormalTok{())}
\end{Highlighting}
\end{Shaded}

\hypertarget{tbl-marble-wl}{}
\begin{longtable}{rrc}
\caption{\label{tbl-marble-wl}Representing marble compositions with `win-loss' plots }\tabularnewline

\toprule
blue\_marbs & white\_marbs & marbles \\ 
\midrule
0 & 4 & <?xml version='1.0' encoding='UTF-8' ?><svg xmlns='http://www.w3.org/2000/svg' xmlns:xlink='http://www.w3.org/1999/xlink' class='svglite' width='58.06pt' height='10.80pt' viewBox='0 0 58.06 10.80'><defs>  <style type='text/css'><![CDATA[    .svglite line, .svglite polyline, .svglite polygon, .svglite path, .svglite rect, .svglite circle {      fill: none;      stroke: #000000;      stroke-linecap: round;      stroke-linejoin: round;      stroke-miterlimit: 10.00;    }    .svglite text {      white-space: pre;    }  ]]></style></defs><rect width='100%' height='100%' style='stroke: none; fill: none;'/><defs>  <clipPath id='cpMC4wMHw1OC4wNnwwLjAwfDEwLjgw'>    <rect x='0.00' y='0.00' width='58.06' height='10.80' />  </clipPath></defs><g clip-path='url(#cpMC4wMHw1OC4wNnwwLjAwfDEwLjgw)'><line x1='4.19' y1='8.91' x2='4.19' y2='4.70' style='stroke-width: 2.13; stroke: #FFFFFF;' /><line x1='7.30' y1='8.91' x2='7.30' y2='4.70' style='stroke-width: 2.13; stroke: #FFFFFF;' /><line x1='10.40' y1='8.91' x2='10.40' y2='4.70' style='stroke-width: 2.13; stroke: #FFFFFF;' /><line x1='13.51' y1='8.91' x2='13.51' y2='4.70' style='stroke-width: 2.13; stroke: #FFFFFF;' /></g></svg> \\ 
1 & 3 & <?xml version='1.0' encoding='UTF-8' ?><svg xmlns='http://www.w3.org/2000/svg' xmlns:xlink='http://www.w3.org/1999/xlink' class='svglite' width='58.06pt' height='10.80pt' viewBox='0 0 58.06 10.80'><defs>  <style type='text/css'><![CDATA[    .svglite line, .svglite polyline, .svglite polygon, .svglite path, .svglite rect, .svglite circle {      fill: none;      stroke: #000000;      stroke-linecap: round;      stroke-linejoin: round;      stroke-miterlimit: 10.00;    }    .svglite text {      white-space: pre;    }  ]]></style></defs><rect width='100%' height='100%' style='stroke: none; fill: none;'/><defs>  <clipPath id='cpMC4wMHw1OC4wNnwwLjAwfDEwLjgw'>    <rect x='0.00' y='0.00' width='58.06' height='10.80' />  </clipPath></defs><g clip-path='url(#cpMC4wMHw1OC4wNnwwLjAwfDEwLjgw)'><line x1='4.19' y1='1.89' x2='4.19' y2='6.10' style='stroke-width: 2.13; stroke: #0000FF;' /><line x1='7.30' y1='8.91' x2='7.30' y2='4.70' style='stroke-width: 2.13; stroke: #FFFFFF;' /><line x1='10.40' y1='8.91' x2='10.40' y2='4.70' style='stroke-width: 2.13; stroke: #FFFFFF;' /><line x1='13.51' y1='8.91' x2='13.51' y2='4.70' style='stroke-width: 2.13; stroke: #FFFFFF;' /></g></svg> \\ 
2 & 2 & <?xml version='1.0' encoding='UTF-8' ?><svg xmlns='http://www.w3.org/2000/svg' xmlns:xlink='http://www.w3.org/1999/xlink' class='svglite' width='58.06pt' height='10.80pt' viewBox='0 0 58.06 10.80'><defs>  <style type='text/css'><![CDATA[    .svglite line, .svglite polyline, .svglite polygon, .svglite path, .svglite rect, .svglite circle {      fill: none;      stroke: #000000;      stroke-linecap: round;      stroke-linejoin: round;      stroke-miterlimit: 10.00;    }    .svglite text {      white-space: pre;    }  ]]></style></defs><rect width='100%' height='100%' style='stroke: none; fill: none;'/><defs>  <clipPath id='cpMC4wMHw1OC4wNnwwLjAwfDEwLjgw'>    <rect x='0.00' y='0.00' width='58.06' height='10.80' />  </clipPath></defs><g clip-path='url(#cpMC4wMHw1OC4wNnwwLjAwfDEwLjgw)'><line x1='4.19' y1='1.89' x2='4.19' y2='6.10' style='stroke-width: 2.13; stroke: #0000FF;' /><line x1='7.30' y1='1.89' x2='7.30' y2='6.10' style='stroke-width: 2.13; stroke: #0000FF;' /><line x1='10.40' y1='8.91' x2='10.40' y2='4.70' style='stroke-width: 2.13; stroke: #FFFFFF;' /><line x1='13.51' y1='8.91' x2='13.51' y2='4.70' style='stroke-width: 2.13; stroke: #FFFFFF;' /></g></svg> \\ 
3 & 1 & <?xml version='1.0' encoding='UTF-8' ?><svg xmlns='http://www.w3.org/2000/svg' xmlns:xlink='http://www.w3.org/1999/xlink' class='svglite' width='58.06pt' height='10.80pt' viewBox='0 0 58.06 10.80'><defs>  <style type='text/css'><![CDATA[    .svglite line, .svglite polyline, .svglite polygon, .svglite path, .svglite rect, .svglite circle {      fill: none;      stroke: #000000;      stroke-linecap: round;      stroke-linejoin: round;      stroke-miterlimit: 10.00;    }    .svglite text {      white-space: pre;    }  ]]></style></defs><rect width='100%' height='100%' style='stroke: none; fill: none;'/><defs>  <clipPath id='cpMC4wMHw1OC4wNnwwLjAwfDEwLjgw'>    <rect x='0.00' y='0.00' width='58.06' height='10.80' />  </clipPath></defs><g clip-path='url(#cpMC4wMHw1OC4wNnwwLjAwfDEwLjgw)'><line x1='4.19' y1='1.89' x2='4.19' y2='6.10' style='stroke-width: 2.13; stroke: #0000FF;' /><line x1='7.30' y1='1.89' x2='7.30' y2='6.10' style='stroke-width: 2.13; stroke: #0000FF;' /><line x1='10.40' y1='1.89' x2='10.40' y2='6.10' style='stroke-width: 2.13; stroke: #0000FF;' /><line x1='13.51' y1='8.91' x2='13.51' y2='4.70' style='stroke-width: 2.13; stroke: #FFFFFF;' /></g></svg> \\ 
4 & 0 & <?xml version='1.0' encoding='UTF-8' ?><svg xmlns='http://www.w3.org/2000/svg' xmlns:xlink='http://www.w3.org/1999/xlink' class='svglite' width='58.06pt' height='10.80pt' viewBox='0 0 58.06 10.80'><defs>  <style type='text/css'><![CDATA[    .svglite line, .svglite polyline, .svglite polygon, .svglite path, .svglite rect, .svglite circle {      fill: none;      stroke: #000000;      stroke-linecap: round;      stroke-linejoin: round;      stroke-miterlimit: 10.00;    }    .svglite text {      white-space: pre;    }  ]]></style></defs><rect width='100%' height='100%' style='stroke: none; fill: none;'/><defs>  <clipPath id='cpMC4wMHw1OC4wNnwwLjAwfDEwLjgw'>    <rect x='0.00' y='0.00' width='58.06' height='10.80' />  </clipPath></defs><g clip-path='url(#cpMC4wMHw1OC4wNnwwLjAwfDEwLjgw)'><line x1='4.19' y1='1.89' x2='4.19' y2='6.10' style='stroke-width: 2.13; stroke: #0000FF;' /><line x1='7.30' y1='1.89' x2='7.30' y2='6.10' style='stroke-width: 2.13; stroke: #0000FF;' /><line x1='10.40' y1='1.89' x2='10.40' y2='6.10' style='stroke-width: 2.13; stroke: #0000FF;' /><line x1='13.51' y1='1.89' x2='13.51' y2='6.10' style='stroke-width: 2.13; stroke: #0000FF;' /></g></svg> \\ 
\bottomrule
\end{longtable}

I'm not overwhelmed by the result. I'll try emojis instead.

\begin{Shaded}
\begin{Highlighting}[]
\DocumentationTok{\#\# Getting the blue circle emoji}
\NormalTok{blue\_marb }\OtherTok{\textless{}{-}}\NormalTok{ emojis }\SpecialCharTok{|\textgreater{}} 
  \FunctionTok{filter}\NormalTok{(}\FunctionTok{str\_detect}\NormalTok{(name, }\StringTok{"blue"}\NormalTok{), }
         \FunctionTok{str\_detect}\NormalTok{(name, }\StringTok{"circle"}\NormalTok{)) }\SpecialCharTok{|\textgreater{}} 
  \FunctionTok{pull}\NormalTok{(emoji)}

\NormalTok{blue\_marb}
\end{Highlighting}
\end{Shaded}

\begin{verbatim}
[1] "🔵"
\end{verbatim}

\begin{Shaded}
\begin{Highlighting}[]
\DocumentationTok{\#\# Getting the white circle emoji}
\NormalTok{white\_marb }\OtherTok{\textless{}{-}}\NormalTok{ emojis }\SpecialCharTok{|\textgreater{}} 
  \FunctionTok{filter}\NormalTok{(}\FunctionTok{str\_detect}\NormalTok{(name, }\StringTok{"white"}\NormalTok{), }
         \FunctionTok{str\_detect}\NormalTok{(name, }\StringTok{"circle"}\NormalTok{)) }\SpecialCharTok{|\textgreater{}} 
  \FunctionTok{pull}\NormalTok{(emoji)}

\NormalTok{white\_marb}
\end{Highlighting}
\end{Shaded}

\begin{verbatim}
[1] "⚪"
\end{verbatim}

\begin{Shaded}
\begin{Highlighting}[]
\FunctionTok{tibble}\NormalTok{(}
  \AttributeTok{blue\_marbs =} \DecValTok{0}\SpecialCharTok{:}\DecValTok{4}\NormalTok{,}
  \AttributeTok{white\_marbs =} \DecValTok{4} \SpecialCharTok{{-}}\NormalTok{ blue\_marbs}
\NormalTok{) }\SpecialCharTok{|\textgreater{}} 
  \FunctionTok{rowwise}\NormalTok{() }\SpecialCharTok{|\textgreater{}} 
  \FunctionTok{mutate}\NormalTok{(}
    \AttributeTok{marbles =} \FunctionTok{list}\NormalTok{(}\FunctionTok{c}\NormalTok{(}\FunctionTok{rep}\NormalTok{(blue\_marb, blue\_marbs), }
                     \FunctionTok{rep}\NormalTok{(white\_marb, white\_marbs)))}
\NormalTok{  ) }\OtherTok{{-}\textgreater{}} 
\NormalTok{  marbles\_emoji}
\end{Highlighting}
\end{Shaded}

\begin{Shaded}
\begin{Highlighting}[]
\NormalTok{marbles\_emoji }\SpecialCharTok{|\textgreater{}} 
  \FunctionTok{gt}\NormalTok{()}
\end{Highlighting}
\end{Shaded}

\hypertarget{tbl-marbles-emoji}{}
\begin{longtable}{rrc}
\caption{\label{tbl-marbles-emoji}Representing marble compositions with emoji }\tabularnewline

\toprule
blue\_marbs & white\_marbs & marbles \\ 
\midrule
0 & 4 & ⚪, ⚪, ⚪, ⚪ \\ 
1 & 3 & 🔵, ⚪, ⚪, ⚪ \\ 
2 & 2 & 🔵, 🔵, ⚪, ⚪ \\ 
3 & 1 & 🔵, 🔵, 🔵, ⚪ \\ 
4 & 0 & 🔵, 🔵, 🔵, 🔵 \\ 
\bottomrule
\end{longtable}

Yes, this is it.

\hypertarget{rerunning-the-sampling}{%
\subsubsection{Rerunning the sampling}\label{rerunning-the-sampling}}

I'll re-run the sampling from
\href{../2023-05-09_02-small-large-worlds/index.qmd}{the previous post}.

\begin{Shaded}
\begin{Highlighting}[]
\NormalTok{sampling\_df }\OtherTok{\textless{}{-}} \ControlFlowTok{function}\NormalTok{(marbles, }
                        \AttributeTok{n =} \DecValTok{1000}\NormalTok{, }
                        \AttributeTok{size =} \DecValTok{3}\NormalTok{, }
                        \AttributeTok{pattern =} \FunctionTok{c}\NormalTok{(blue\_marb, white\_marb, blue\_marb))\{}
\NormalTok{  sampling\_tibble }\OtherTok{\textless{}{-}} \FunctionTok{tibble}\NormalTok{(}\AttributeTok{samp =} \DecValTok{1}\SpecialCharTok{:}\NormalTok{n)   }
\NormalTok{  sampling\_tibble }\SpecialCharTok{|\textgreater{}} 
    \FunctionTok{mutate}\NormalTok{(}
      \AttributeTok{chosen =} \FunctionTok{map}\NormalTok{(samp, }
                   \SpecialCharTok{\textasciitilde{}}\FunctionTok{sample}\NormalTok{(marbles, }
                           \AttributeTok{size =} \DecValTok{3}\NormalTok{, }
                           \AttributeTok{replace =}\NormalTok{ T)),}
      \AttributeTok{match =} \FunctionTok{map\_lgl}\NormalTok{(chosen, }
                      \SpecialCharTok{\textasciitilde{}}\FunctionTok{all}\NormalTok{(.x }\SpecialCharTok{==}\NormalTok{ pattern))                 }
\NormalTok{    ) }\SpecialCharTok{|\textgreater{}} 
    \FunctionTok{summarise}\NormalTok{(}\AttributeTok{prop\_match =} \FunctionTok{mean}\NormalTok{(match))}\OtherTok{{-}\textgreater{}}                         
\NormalTok{    sampling\_tibble}
  \FunctionTok{return}\NormalTok{(sampling\_tibble)}
\NormalTok{\}}
\end{Highlighting}
\end{Shaded}

\begin{Shaded}
\begin{Highlighting}[]
\NormalTok{marbles\_emoji }\SpecialCharTok{|\textgreater{}} 
 \FunctionTok{ungroup}\NormalTok{() }\SpecialCharTok{|\textgreater{}} 
  \FunctionTok{mutate}\NormalTok{(}
    \AttributeTok{prob =} \FunctionTok{map}\NormalTok{(marbles, }\SpecialCharTok{\textasciitilde{}}\FunctionTok{sampling\_df}\NormalTok{(.x, }\AttributeTok{n =} \DecValTok{10000}\NormalTok{))}
\NormalTok{  ) }\SpecialCharTok{|\textgreater{}} 
  \FunctionTok{unnest}\NormalTok{(prob) }\SpecialCharTok{|\textgreater{}} 
  \FunctionTok{mutate}\NormalTok{(}\AttributeTok{norm\_probs =}\NormalTok{ prop\_match}\SpecialCharTok{/}\FunctionTok{sum}\NormalTok{(prop\_match))}\OtherTok{{-}\textgreater{}}
\NormalTok{  marble\_probs}
\end{Highlighting}
\end{Shaded}

I want to label the column of probabilities with the key sequence.

\begin{Shaded}
\begin{Highlighting}[]
\NormalTok{key\_seq }\OtherTok{\textless{}{-}} \FunctionTok{str\_glue}\NormalTok{(}\StringTok{"\{blue\_marb\}, \{white\_marb\}, \{blue\_marb\}"}\NormalTok{)}
\end{Highlighting}
\end{Shaded}

\texttt{gtExtras::gt\_plt\_bar\_pct()} will plot a bar chart within the
table.

\begin{Shaded}
\begin{Highlighting}[]
\NormalTok{marble\_probs }\SpecialCharTok{|\textgreater{}} 
  \FunctionTok{select}\NormalTok{(marbles, norm\_probs) }\SpecialCharTok{|\textgreater{}} 
  \FunctionTok{mutate}\NormalTok{(}\AttributeTok{norm\_probs =}\NormalTok{ norm\_probs }\SpecialCharTok{*} \DecValTok{100}\NormalTok{) }\SpecialCharTok{|\textgreater{}} 
  \FunctionTok{gt}\NormalTok{() }\SpecialCharTok{|\textgreater{}} 
  \FunctionTok{cols\_label}\NormalTok{(}
    \AttributeTok{norm\_probs =} \FunctionTok{str\_glue}\NormalTok{(}\StringTok{"p(marbles | \{key\_seq\})"}\NormalTok{)}
\NormalTok{  ) }\SpecialCharTok{|\textgreater{}} 
  \FunctionTok{gt\_plt\_bar\_pct}\NormalTok{(norm\_probs, }
                 \AttributeTok{scaled =}\NormalTok{ T, }
                 \AttributeTok{fill =} \StringTok{"steelblue"}\NormalTok{)}
\end{Highlighting}
\end{Shaded}

\hypertarget{tbl-upate1}{}
\begin{longtable}{cl}
\caption{\label{tbl-upate1}Probability of each marble composition given (🔵, ⚪️, 🔵) samples with
replacement. }\tabularnewline

\toprule
marbles & p(marbles | 🔵, ⚪, 🔵) \\ 
\midrule
⚪, ⚪, ⚪, ⚪ & <div style='flex-grow:1;margin-left:8px;background:#e1e1e1;'><div style='background:steelblue;width:0%;height:16px;'></div></div> \\ 
🔵, ⚪, ⚪, ⚪ & <div style='flex-grow:1;margin-left:8px;background:#e1e1e1;'><div style='background:steelblue;width:14.9532710280374%;height:16px;'></div></div> \\ 
🔵, 🔵, ⚪, ⚪ & <div style='flex-grow:1;margin-left:8px;background:#e1e1e1;'><div style='background:steelblue;width:39.8968739929101%;height:16px;'></div></div> \\ 
🔵, 🔵, 🔵, ⚪ & <div style='flex-grow:1;margin-left:8px;background:#e1e1e1;'><div style='background:steelblue;width:45.1498549790525%;height:16px;'></div></div> \\ 
🔵, 🔵, 🔵, 🔵 & <div style='flex-grow:1;margin-left:8px;background:#e1e1e1;'><div style='background:steelblue;width:0%;height:16px;'></div></div> \\ 
\bottomrule
\end{longtable}

There we go!

\hypertarget{with-the-bayesian-update}{%
\subsubsection{With the Bayesian
Update}\label{with-the-bayesian-update}}

\begin{Shaded}
\begin{Highlighting}[]
\NormalTok{marble\_probs }\SpecialCharTok{|\textgreater{}} 
  \FunctionTok{mutate}\NormalTok{(}\AttributeTok{new\_prob =}\NormalTok{ blue\_marbs}\SpecialCharTok{/}\FunctionTok{sum}\NormalTok{(blue\_marbs),}
         \AttributeTok{multiplied =}\NormalTok{ norm\_probs }\SpecialCharTok{*}\NormalTok{ new\_prob,}
         \AttributeTok{norm\_new =}\NormalTok{ multiplied}\SpecialCharTok{/}\FunctionTok{sum}\NormalTok{(multiplied)) }\SpecialCharTok{|\textgreater{}} 
  \FunctionTok{select}\NormalTok{(marbles, norm\_probs, norm\_new) }\SpecialCharTok{|\textgreater{}} 
  \FunctionTok{mutate}\NormalTok{(}\AttributeTok{norm\_probs =}\NormalTok{ norm\_probs }\SpecialCharTok{*} \DecValTok{100}\NormalTok{,}
         \AttributeTok{norm\_new =}\NormalTok{ norm\_new }\SpecialCharTok{*} \DecValTok{100}\NormalTok{) }\SpecialCharTok{|\textgreater{}} 
  \FunctionTok{gt}\NormalTok{() }\SpecialCharTok{|\textgreater{}} 
  \FunctionTok{cols\_label}\NormalTok{(}
    \AttributeTok{norm\_probs =} \FunctionTok{str\_glue}\NormalTok{(}\StringTok{"p(marbles | \{key\_seq\})"}\NormalTok{),}
    \AttributeTok{norm\_new =} \FunctionTok{str\_glue}\NormalTok{(}\StringTok{"after \{blue\_marb\}"}\NormalTok{)}
\NormalTok{  ) }\SpecialCharTok{|\textgreater{}} 
  \FunctionTok{gt\_plt\_bar\_pct}\NormalTok{(norm\_probs, }
                 \AttributeTok{scaled =}\NormalTok{ T, }
                 \AttributeTok{fill =} \StringTok{"steelblue"}\NormalTok{) }\SpecialCharTok{|\textgreater{}} 
  \FunctionTok{gt\_plt\_bar\_pct}\NormalTok{(norm\_new, }
                 \AttributeTok{scaled =}\NormalTok{ T, }
                 \AttributeTok{fill =} \StringTok{"steelblue"}\NormalTok{) }\SpecialCharTok{|\textgreater{}} 
  \DocumentationTok{\#\# Necessary to get the percent bars}
  \DocumentationTok{\#\# to be equal length.}
  \FunctionTok{cols\_width}\NormalTok{(}\DecValTok{2} \SpecialCharTok{\textasciitilde{}} \FunctionTok{px}\NormalTok{(}\DecValTok{200}\NormalTok{),}
             \DecValTok{3} \SpecialCharTok{\textasciitilde{}} \FunctionTok{px}\NormalTok{(}\DecValTok{200}\NormalTok{))}
\end{Highlighting}
\end{Shaded}

\hypertarget{tbl-upate2}{}
\begin{longtable}{cll}
\caption{\label{tbl-upate2}Probability of each marble composition given an additional (🔵) sample }\tabularnewline

\toprule
marbles & p(marbles | 🔵, ⚪, 🔵) & after 🔵 \\ 
\midrule
⚪, ⚪, ⚪, ⚪ & <div style='flex-grow:1;margin-left:8px;background:#e1e1e1;'><div style='background:steelblue;width:0%;height:16px;'></div></div> & <div style='flex-grow:1;margin-left:8px;background:#e1e1e1;'><div style='background:steelblue;width:0%;height:16px;'></div></div> \\ 
🔵, ⚪, ⚪, ⚪ & <div style='flex-grow:1;margin-left:8px;background:#e1e1e1;'><div style='background:steelblue;width:14.9532710280374%;height:16px;'></div></div> & <div style='flex-grow:1;margin-left:8px;background:#e1e1e1;'><div style='background:steelblue;width:6.49587008259835%;height:16px;'></div></div> \\ 
🔵, 🔵, ⚪, ⚪ & <div style='flex-grow:1;margin-left:8px;background:#e1e1e1;'><div style='background:steelblue;width:39.8968739929101%;height:16px;'></div></div> & <div style='flex-grow:1;margin-left:8px;background:#e1e1e1;'><div style='background:steelblue;width:34.6633067338653%;height:16px;'></div></div> \\ 
🔵, 🔵, 🔵, ⚪ & <div style='flex-grow:1;margin-left:8px;background:#e1e1e1;'><div style='background:steelblue;width:45.1498549790525%;height:16px;'></div></div> & <div style='flex-grow:1;margin-left:8px;background:#e1e1e1;'><div style='background:steelblue;width:58.8408231835363%;height:16px;'></div></div> \\ 
🔵, 🔵, 🔵, 🔵 & <div style='flex-grow:1;margin-left:8px;background:#e1e1e1;'><div style='background:steelblue;width:0%;height:16px;'></div></div> & <div style='flex-grow:1;margin-left:8px;background:#e1e1e1;'><div style='background:steelblue;width:0%;height:16px;'></div></div> \\ 
\bottomrule
\end{longtable}



\end{document}
